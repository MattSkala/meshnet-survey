\documentclass[conference,compsoc]{IEEEtran}
\usepackage[utf8]{inputenc}
\usepackage{cite}

\usepackage[hyphens]{url}
\usepackage[hidelinks]{hyperref}
\hypersetup{breaklinks=true}


\begin{document}
 
\title{A Survey on Smartphone Ad Hoc Networks}

\author{\IEEEauthorblockN{Matouš Skála}
\IEEEauthorblockA{Delft University of Technology\\
Email: M.Skala@student.tudelft.nl}}

\maketitle

\begin{abstract}
Over the past few years, smartphones have become powerful devices that come equipped with numerous network connectivity modules. In addition to Bluetooth BR/EDR and Bluetooth Low Energy, most recent devices also come with support for Wi-Fi Direct and Wi-Fi Aware standards to facilitate the interconnection with nearby devices over longer distances and with higher bandwidth. 

In this survey, we discuss the current implementation of various connectivity APIs in Android OS and explore possibilities of deploying a large-scale smartphone ad hoc network. Additionally, we design a proof of concept Android application to demonstrate the feasibility of mesh networking within the constraints imposed by Android OS.

\end{abstract}

\begin{IEEEkeywords}mesh networks, mobile ad hoc networks\end{IEEEkeywords}

\section{Introduction}

The increasing popularity of smartphones and novel wireless networking technologies opens up possibilities to a whole new range of applications that can communicate without the need for the Internet connection. Prospective use cases are ranging from proximity-based social networks, infrastructure-less communication with \textit{Internet of Things (IoT)} devices, student attendance tracking, to communication between attendees during music festivals or protests, where the cellular network is either overloaded or censored \cite{forbes:hk}. 

Starting from Android 4.0, \textit{Wi-Fi Direct} \cite{android:wifip2p}, also referred to as Wi-Fi peer to peer, is directly supported by the system, allowing device to device Wi-Fi communication without an additional \textit{access point (AP)}. From Android 8.0, the \textit{Wi-Fi Aware} \cite{android:wifiaware} standard, also known as \textit{Neighbor Awareness Networking (NAN)}, has been supported, allowing to automatically form clusters of nearby devices.

Wi-Fi generally offers a higher range of coverage and bandwidth than \textit{Bluetooth}, so it might be more suitable for data-intensive applications such as photo or video sharing. On the other hand, Bluetooth is supported on a wider variety of devices and especially \textit{Bluetooth Low Energy (BLE)} \cite{android:ble} can result in significantly lower battery consumption, thus it is more suitable for transferring small amounts of data.

This survey is structured as follows. In Section \ref{wirelesstech}, we first explore features of all previously mentioned wireless communication protocols. In Section \ref{android}, we discuss their implementation and possible limitations within Android OS. In Section \ref{applications}, we explore different Android applications taking advantage of mesh networking for censorship resilient messaging. Finally, in Section \ref{poc} we present our proof of concept for deploying an ad hoc network with multi-hop routing on Android. Section \ref{conclusion} concludes this work.

\section{Wireless Communication Technologies} \label{wirelesstech}

\subsection{Bluetooth}


\subsection{Bluetooth Low Energy}
\subsection{WiFi Direct}
\subsection{WiFi Aware}

\section{Implementation in Android OS} \label{android}
\subsection{BluetoothManager}
\subsection{WifiP2pManager}
\subsection{WifiAwareManager}
\subsection{Nearby Connections API}

\section{Applications} \label{applications}
\subsection{Briar}
\subsection{Bridgefy}
\subsection{FireChat}
\subsection{The Serval Mesh}
\subsection{B.A.T.M.A.N.}

\section{Proof of Concept} \label{poc}
\section{Conclusion} \label{conclusion}

\bibliographystyle{IEEEtran}
\bibliography{IEEEabrv,mybibfile}

\end{document}
